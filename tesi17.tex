\documentclass[a4paper,10pt]{article}

%\usepackage{ucs}
\usepackage[utf8]{inputenc}
%\usepackage[latin1]{inputenc} %non gli piace
\usepackage{amsmath}
\usepackage{amsfonts}
\usepackage{amssymb}	
\usepackage{bm}
%\usepackage{babel}
\usepackage{fontenc}
\usepackage{graphicx}

\usepackage[dvips]{hyperref}

\date{11/4/2019}

\begin{document}
 \section*{Chapter 2}
 In this chapter symmetries and action principles will be analyzed by treating them with classical field theory.
 \section*{Lagrangian function in Classical Field Theory}
 In classical mechanics, a system can be described by a function of space-time coordinates, and their time derivatives, the Lagrangian.
 This function gives information about the dynamics of the system.\\
 The same appends in field theory.\\
 Let's consider a system of many fields $\phi^i$ in an n-dimensional Minkowsky spae-time.\\
 The dynamics of this system can still be described with Lagrangian formalism but it need to be generalized.
 To do so, we must introduce the concept of Lagrangian Density, which is defined as:
 \begin{equation}
  L=\int{d^{n-1}x\;\mathcal{L} }
 \end{equation}
 Where L is the total Lagrangian fuction and 
 \begin{equation}
  \mathcal{L}= \mathcal{L} \left(\phi^i(x^\mu),\frac{\partial\phi_i}{\partial x^\mu} \right)
 \end{equation}
 is the Lagrangian density,which depends on the fields and their derivatives.\\
 Like in the classical case, the equation of motion can be found by minimizing the variation of the action which is now defined as:
 \begin{equation}
  \mathcal{S}=\int{dx^0\;L } = \int{d^nx\;\mathcal{L} }
 \end{equation}
 \\
 \\
 \section*{Symmetries}
 Consider a set of fields $\phi^i(x)$ which satisfy the equation of motion.
 A general symmetry of the system is a mapping of the configuration space, $\phi^i(x)\rightarrow\phi^{'i}(x)$,
 with the property that if the original field configuration $\phi^i(x)$ is a solution of the equations
 of motion, then also the transformed configuration $\phi^{'i}(x)$ is. 
 For scalar fields and more other systems, we can restrict attention to those symmetries that leave the action invariant.\\
 Thus we require that the transformation has the following property:
 \begin{equation}
  S(\phi^i)=S(\phi^{'i})
 \end{equation}
 There are two main types of symmetries: spacetime symmetries, which involve a motion in the spacetime, and internal symmetries, which do not.
 \section{Internal Symmetries}
 Considering a linearly realized internal symmetry under an arbitrary connected Lie group G, we can see a general element of the algebra as a matrix $\Theta$ which is a superposition of the genertors $t_A$ and real parameters $\theta^A$
 \begin{equation}
  \Theta= t^A\theta_A
 \end{equation}
  So an element of the group is represented by the matrix exponential
  \begin{equation}
   U\left(\varTheta \right) = e^{-\varTheta} = e^{-t^A\theta_A}
  \end{equation}
 This kind of transformation acts by matrix multiplication on the field
 \begin{equation}
  \phi^i(x) \rightarrow \phi^{'i}(x) = U\left(\varTheta \right)^i_j \phi^j(x)
 \end{equation}
 To obtain the field variation we shall focus on infinitesimal transformations that are defined as the truncation of the power series of the group's element U($\Theta$) at the first order in $\Theta$.\\
 Thus we have:
 \begin{equation}
  \delta\phi=-\Theta\phi
 \end{equation}
As we can see, an internal symmetry is a trasformation which acts only on the field, leaving the spacetime points invariant.
An example of this kind of symmetry is the SO(n) one.


\section{Spacetime Symmetries}
The aim of this section is to analyze spacetime symmetries and how they act on fields.
To do so, we will consider an example which is very importan in General Relativity:
The Lorentz and Poincaré symmetry.\\
In an n-dimensional flat spacetime, the Lorentz group is defined as the group of homogeneus linear transformations of coordinates:
\begin{equation}
 x^{\mu} = \Lambda^\mu_\nu x^{'\nu}
\end{equation}
This transformations preserve the Minkowski norm, so we require the property:
\begin{equation}
 x^\mu\eta_{\mu\nu}x^\nu=x^{'\mu}\eta_{\mu\nu}x^{'\nu}
\end{equation}
The Poincaré group is the Lorentz group plus a global translation, so that we obtain:
\begin{equation}
 x^{'\mu}=\Lambda^{-1\mu}_\nu(x^{\nu}-a^{\nu})
\end{equation}
The property of norm preservation leads to the defining property of the $\Lambda$ matrices:
\begin{equation}
 \Lambda^\mu_\rho\eta_{\mu\nu}\Lambda^{\nu}_\sigma=\eta_{\rho\sigma}
\end{equation}
The action of these matrices on a scalar field $\phi^i(x)$ is
\begin{equation}
 \phi^i(x)\rightarrow \phi^{'i}(x)= \phi^i(\Lambda x)
\end{equation}
It is usefull to analyze an infinitesimal transformation. To do so, we shall introduce differential operators that implement the coordinate change in the $[\alpha\beta]$ 2-plane. 
\begin{equation}
 L_{[\alpha\beta]}= x_\alpha \partial_\beta - x_\beta \partial_\alpha
\end{equation}
Thus we have a realization of the Lie algebra of the Lorentz group acting as differential operators on functions.\\
An element of the group can be seen as the operator
\begin{equation}
 U(\Lambda)=e^{-\frac{1}{2} \lambda^{\alpha\beta}L_{[\alpha\beta]}}
\end{equation}
The action of finite Lorentz transformations on scalar fields can be written in fuction of this operator
\begin{equation}
 \phi^i(x)\rightarrow\phi^{'i}(x) = U(\Lambda)\phi^i(x) = \phi^i(\Lambda x)
\end{equation}
In the end we see that both Lorentz and internal symmetries are implemented by linear operators acting on the classical fields, a differential operator for Lorentz and a matrix
operator for internal and both operators depend on group parameters in the same way.\\
Now, we want to study the Poincaré transformations, which are given by the Lorentz's ones plus a spacetime translation, defined as
\begin{equation}
 \phi^i(x)\rightarrow\phi^{'i}(x)= \phi^i(x+a) = U(a) \phi^i(x)
\end{equation}
Thus we have
\begin{equation}
 U(a)=e^{a^\mu P_\mu}
\end{equation}
Where 
\begin{equation}
 P_\mu= \partial_\mu= \frac{\partial}{\partial x_\mu}
\end{equation}
So, the finite transformations of the Poincaré group are implemented by the operator 
 $U(a,\Lambda)=U(a)U(\Lambda)$
that acts in the following way on the fields
\begin{equation}
 \phi^i(x)\rightarrow\phi^{'i}(x)=U(a,\Lambda) \phi^i(x) =\phi^i(\Lambda x +a)
\end{equation}



\section*{Noether's Theorem}
In this section we want to analyze, using the Noether's Theorems, the relation between global symmetries and conserved charges. We also want to define the canonical energy-momentum tensor as the conserved Noether current associated with the invariance of the action under certain spacetime transformations.\\
Let's consider a generic field $\phi(x)$ and its action $ S(\phi)$ which is defined as
\begin{equation}
 S(\phi)=\int { d^dx\; \mathcal{L} \left( \phi, \partial_\mu \phi \right)}
\end{equation}
To find the equation of motion we should minimize the action, thus, under infinitesimal variations of the field $\delta \phi$ we obtain
\begin{equation}
 \delta S = \int{ d^dx\;\delta \mathcal{L}}= \int{ d^dx\;( \frac{\partial \mathcal{L}}{\partial \phi} \delta \phi + \frac{\partial \mathcal{L}}{\partial \partial_\mu \phi} \delta \partial_\mu \phi)}
\end{equation}
So, to obtain the least action we require that $\delta S =0$. This lead us to the Euler-Lagrange equations of motion
\begin{equation}
 \frac{\partial \mathcal{L}}{\partial \phi}\;-\partial_\mu \left(\frac{\partial \mathcal{L}}{\partial \partial_\mu \phi}\right)=0
\end{equation}
If we now consider a generic infinitesimal symmetry variation of the field, defined as
\begin{equation}
 \delta \phi(x)= \varsigma^A \; \Delta_A \phi(x)
\end{equation}
where we assume that $\varsigma^A$ are constant parameters.\\
As a symmetry, this transformation leaves the action invariant, thus the variation of the Lagrangian density must be a total derivative $ \delta \mathcal{L}= \varsigma^A \partial_\mu K^\mu_A$.\\
The variation of the Lagrangian density in detail is
\begin{equation}
 \delta \mathcal{L}= \, \varsigma^A \left(\frac{\partial \mathcal{L}}{\partial \phi}\Delta_A\phi +\frac{\partial \mathcal{L}}{\partial \partial_\mu \phi}\partial_\mu \Delta_A \phi \right)= \varsigma^A \partial_\mu K^\mu_A
\end{equation}
Using the Euler-Lagrange equations, we can rewrite this variation as
\begin{equation}
 \partial_\mu J^\mu_A=0
\end{equation}
Which holds for all the solutions of the equations of motion.
This means that $J^\mu_A$ is the associated conserved Noether current, thus we have:
\begin{equation}
 J^\mu_A=\,-\frac{\partial \mathcal{L}}{\partial \partial_\mu \phi} \Delta_A\phi \, +\, K^\mu_A
\end{equation}
By considering $\varsigma^A$ as varying parameters $\varsigma^A(x)$ we obtain
\begin{equation}
\begin{split}
  &\delta S= \int{ d^dx \,\left[ \frac{\partial \mathcal{L}}{\partial \phi}\,\varsigma^A \Delta_A\phi + \frac{\partial \mathcal{L}}{\partial \partial_\mu \phi}\partial_\mu(\varsigma^A \Delta_A\phi)\right]}\\
  &=\, \int{ d^dx \,\left[\varsigma^A \partial_\mu K^\mu_A+\,(\partial_\mu\varsigma^A) \frac{\partial \mathcal{L}}{\partial \partial_\mu \phi} \Delta_A \phi \right]}\\
  &=\, -\int{d^dx\,(\partial_\mu\varsigma^A)\,J^\mu_A}
\end{split}                        
\end{equation}
For each Noether current one can define an integrated conserved charge, which is a constant of motion, by taking the integral of the time component of the current $J^0_A$ over a family of constant time surfaces that are foliating our flat spacetime.\\
Thus we have
\begin{equation}
 Q_A=\,\int{d^{(d-1)}x\,J^0_A(\overrightarrow{x},t)}
\end{equation}
To see which charge is associated to internal and spacetime transformations, one can take as an example a massless real scalar field. The lagrangian density and equation of motion are
\begin{equation}
 \begin{split}
  &\mathcal{L}(\phi)=\frac{1}{2}\,\partial_\mu \phi\, \partial^\mu \phi\\
  &\partial^2 \phi=0
 \end {split}
\end{equation}
For internal symmetry,where $\delta\phi=\,t^a\theta_A\phi$, we have
\begin{equation}
 J^\mu_A=\,-\partial^\mu\phi\,t^A\phi
\end{equation}
While for spacetime translations, the index A of the current is replaced by a vector $\nu$ and the conserved current is the canonical energy-momentum tensor
\begin{equation}
 T^\mu_\nu=\, \partial^\mu\phi\,\partial_\nu\phi\,+\,\delta^\mu_\nu\mathcal{L}
\end{equation}
By adding Lorentz transformations, the index A becomes an antisymmetric pair $\rho\sigma$
and $J^\mu_A$ becomes
\begin{equation}
 M^\mu_{[\rho \sigma]}=-x_\rho T^\mu_\sigma \, + \, x_\sigma T^\mu_\rho
\end{equation}
Thus the conserved charges for those symmetry, called $T_A$ for the internal one, $P_\mu$ for translations and $M_{[\rho \sigma]}$ for Lorentz translations are
\begin{equation}
 \begin{align}
  &T_A=\int{d^{(d-1)}\overrightarrow{x}\,J^0_A},\\
  &P_\mu=\int{d^{(d-1)}\overrightarrow{x}\,T^0_\mu},\\
  &M_{[\rho \sigma]}=\int{d^{[d-1]}\overrightarrow{x}\,M^0_{[\rho \sigma]}}
 \end{align}
\end{equation}
In general the symmetry currents obtained by the Noether procedure are not unique.
In fact they can be modified by adding terma like $\Delta\,J^\mu_a$ which are identically conserved.
Neither Noether charges are changed by the added term since $\Delta\,J^0_A$ involves total spatial derivatives.





\end{document}
